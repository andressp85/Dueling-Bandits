\documentclass{article}


\RequirePackage{authblk}
\RequirePackage{xargs}
\RequirePackage[OT1]{fontenc}
\RequirePackage{amsthm,amsmath,txfonts,bm,pifont,graphicx,bbm,enumitem}
\RequirePackage[numbers]{natbib}
\RequirePackage[colorlinks,citecolor=blue,urlcolor=blue]{hyperref}
\RequirePackage[utf8]{inputenc}
\RequirePackage[ruled,vlined]{algorithm2e}
\SetKwHangingKw{KwParameters}{parameters}
\SetKwHangingKw{KwInit}{initialization}
\SetKwHangingKw{KwIn}{input}


\numberwithin{equation}{section}
\theoremstyle{plain}
\newtheorem{theorem}{Theorem}[section]
\newtheorem{proposition}{Proposition}
\newtheorem{corollary}{Corollary}
\newtheorem{lemma}{Lemma}

\theoremstyle{plain}
\newtheorem{definition}{Definition}

\theoremstyle{plain}
\newtheorem{convention}{Convention}

\theoremstyle{remark}
\newtheorem{remark}{Remark}

\theoremstyle{remark}
\newtheorem{example}{Example}

%\setlength{\parskip}{6pt}

\newcommandx{\vt}[2][2=T]{\mathrm{v}_{#2}\{#1\}}
\newcommand{\1}{\ensuremath{\mathbbm{1}}}
\newcommand{\rme}{\mathrm{e}}
\newcommand{\N}{\mathbb{N}}
\newcommand{\Z}{\mathbb{Z}}
\newcommand{\R}{\mathbb{R}}
\newcommand{\E}{\mathbb{E}}
\newcommand{\PP}{\mathbb{P}}
\newcommand{\btheta}{\bm\theta}
\newcommand{\Dfrac}[2]{{\displaystyle \frac{#1}{#2}}}
\newcommand{\rmd}{\mathrm{d}}
\setlength{\parindent}{0in}

%%% parenthesage
\newcommand{\pa}[1]{\left(#1\right)}
\newcommand{\cro}[1]{\left[#1\right]}
\newcommand{\ac}[1]{\left\{#1\right\}}

\renewcommand{\liminf}{\mathop{\mathrm{liminf}}}
%%%
% notation moments Z
\def\m{m}
\def\mphi{\phi}


\newenvironment{hyp}[1]{
\begin{enumerate}[label={\sf(\textbf{#1}-\arabic*)},resume=hyp#1]\begin{sf}}
{\end{sf}\end{enumerate}}

\begin{document}

\title{Dueling}


%\author[1]{Christophe Giraud \thanks{christophe.giraud@math.u-psud.fr}}
%\author[2]{Fran\c{c}ois Roueff\thanks{francois.roueff@telecom-paristech.fr}}
%\author[2]{Andres Sanchez-Perez\thanks{andres.sanchez-perez@telecom-paristech.fr}}
%\affil[1]{Universit\'e Paris Sud ; D\'epartement de Math\'ematiques}
%\affil[2]{Institut Mines-T\'el\'ecom ; T\'el\'ecom ParisTech ; CNRS LTCI}

\maketitle

We define the set $T_{p}=\{2^{p-1},\ldots,2^{p}-1\}$. 
\vspace{0.3cm}

\begin{algorithm}[H] 
\caption{Improved Doubler}
\KwInit{$x_{1}$ fixed in $X$, $\mathcal{L}=\{x_{1}\}$, $\hat{f}_{0}=0$}
$t\gets 1$\;
$p\gets 1$\;
\While{true}{
\For{$j=1$ \KwTo $2^{p-1}$}{
choose $x_{t}$ uniformly from $\mathcal{L}$\;
$y_{t} \gets \textrm{advance}(S)$\;
$\textrm{play}\left(x_{t};y_{t}\right)$, observe choice $b_{t}$\;
$\textrm{feedback}\left(S; b_{t}+\hat{f}_{p-1}\right)$\;
$t\gets t+1$\;
}
$\mathcal{L}$ the multi-set of arms played as $y_{t}$ in the last for-loop\;
$\hat{f}_{p}\gets \hat{f}_{p}+\sum_{s\in T_{p}}b_{s}/2^{p-1}-1/2$\;
$p\gets p+1$\;
}
\end{algorithm}
\vspace{0.3cm}

Observe that if $t\in T_{p}$

\begin{eqnarray*}
\E\left[b_{t}\left|\left\{y_{s},s\in T_{p-1}\right\}, y_{t}\right.\right] &=& \sum_{s\in T_{p-1}}\Dfrac{\mu\left(y_{t}\right)-\mu\left(y_{s}\right)+1}{2^{p-1}} = \Dfrac{\mu\left(y_{t}\right)+1}{2}-\sum_{s\in T_{p-1}}\Dfrac{\mu\left(y_{s}\right)}{2^{p-1}}\;,
\end{eqnarray*}

and that

\begin{eqnarray*}
\E\left[\sum_{s\in T_{p-1}}b_{s}/2^{p-2}-1/2\left|\bigcup_{r=p-2}^{p-1}\left\{y_{s},s\in T_{r}\right\}\right.\right] &=& \sum_{s\in T_{p-1}}\Dfrac{\mu\left(y_{s}\right)}{2^{p-1}}-\sum_{s\in T_{p-2}}\Dfrac{\mu\left(y_{s}\right)}{2^{p-2}}\;.
\end{eqnarray*}

Using the recurrence defining $\hat{f}_{p}$ we obtain

\begin{eqnarray*}
\E\left[b_{t}+\hat{f}_{p-1}\left|x_{1},\bigcup_{r=1}^{p-1}\left\{y_{s},s\in T_{r}\right\}, y_{t}\right.\right] &=& \Dfrac{\mu\left(y_{t}\right)-\mu\left(x_{1}\right)+1}{2}\;.
\end{eqnarray*}

Since the above right term is $\sigma(x_{1},y_{t})$- measurable we conclude that

\begin{eqnarray*}
\E\left[b_{t}+\hat{f}_{p-1}\left|x_{1},y_{t}\right.\right] &=& \Dfrac{\mu\left(y_{t}\right)-\mu\left(x_{1}\right)+1}{2}\;.
\end{eqnarray*}
 

\bibliographystyle{plain}
\bibliography{allbib}

\end{document}
